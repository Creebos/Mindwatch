\documentclass[conference]{IEEEtran}
\IEEEoverridecommandlockouts

\usepackage{cite}
\usepackage{amsmath,amssymb,amsfonts}
\usepackage{algorithmic}
\usepackage{graphicx}
\usepackage{textcomp}
\usepackage{xcolor}
\usepackage{hyperref}
\def\BibTeX{{\rm B\kern-.05em{\sc i\kern-.025em b}\kern-.08em
    T\kern-.1667em\lower.7ex\hbox{E}\kern-.125emX}}

\begin{document}

\title{MindWatch: Keeping Tech Minds in Balance\\
{\footnotesize \textsuperscript{}}
}

\author{\IEEEauthorblockN{Riccardo Dennis William}
\IEEEauthorblockA{\textit{Department of Information System} \\
\textit{Hanyang University}\\
Seoul, South Korea \\
riccaden@students.zhaw.ch}
\and
\IEEEauthorblockN{Meyer Yves}
\IEEEauthorblockA{\textit{Department of Information System} \\
\textit{Hanyang University}\\
Seoul, South Korea \\
meyeryve@students.zhaw.ch}
\and
\IEEEauthorblockN{Wu Yunjie}
\IEEEauthorblockA{\textit{Department of Information System} \\
\textit{Hanyang University}\\
Seoul, South Korea \\
19855420423@163.com}
\and
\IEEEauthorblockN{Pereira Leandro}
\IEEEauthorblockA{\textit{Department of Information System} \\
\textit{Hanyang University}\\
Seoul, South Korea \\
pereilea@students.zhaw.ch}
}

\maketitle

\begin{abstract}
    This project aims to develop a comprehensive web-based
    platform designed to assess the mental health of employees in
    the tech industry through AI-driven analysis of survey
    responses and performance data. The platform will provide
    valuable insights into employee well-being, functioning as an
    early warning system to detect potential mental health concerns
    before they escalate. By offering a dashboard tailored for HR
    professionals, the system enables data-driven decision-making,
    allowing organizations to proactively support employee mental
    health. A key aspect of the platform is the secure handling of
    survey data, ensuring anonymity and confidentiality, which
    fosters trust and encourages widespread adoption among
    employees.
\end{abstract}

\begin{IEEEkeywords}
    mental health, AI-driven analysis, employee well-being, tech 
    industry, HR dashboard, data security, survey automation, burnout 
    detection, stress management, predictive analytics, human resources,
     data-driven decision-making.
\end{IEEEkeywords}


\begin{table}[htbp]
    \caption{Role Assignments (First half of Project)}
    \centering
    \begin{tabular}{|p{1.8cm}|p{0.9cm}|p{4.8cm}|}
        \hline
        \textbf{Roles} & \textbf{Name} & \textbf{Tasks} \\ \hline
        User & Yunjie & Explaining use-cases and testing demo \\ \hline
        Customer & Leandro & Change use-cases during development \\ \hline
        Software Developer & Leandro, Yunjie, Yves, Dennis & Implementing software and creating model \\ \hline
        Development Manager & Yves, Dennis & Prioritize tasks and organize teams \\ \hline
    \end{tabular}
    \label{tab:role_assignments}
\end{table}


\section{Introduction}
The tech industry is known for its high-pressure work environment, 
often leading to mental health challenges among employees. 
Early identification of potential mental health risks and timely 
support can significantly improve employee well-being and overall 
productivity. However, many organizations lack consistent, 
confidential tools to monitor and assess mental health across their 
workforce. This project seeks to address this gap by developing a 
scalable, AI-powered solution for continuous mental health monitoring.

Tech professionals frequently face long hours, tight
deadlines, and high-stakes projects, all of which can
contribute to stress, burnout, and other mental health issues.
Without proper monitoring tools, organizations risk higher
turnover rates, decreased productivity, and rising healthcare
costs. The client, LG, requires a system that will enable HR
departments to identify early signs of mental health challenges
before they escalate into more severe problems.\newline

While there are existing platforms such as Headspace and
Calm, which focus on mental well-being, few solutions offer
continuous mental health monitoring specifically tailored to
the tech industry. Furthermore, many available tools lack
integration with company data or actionable insights for HR
professionals to drive meaningful interventions. Our platform
aims to fill this gap by providing continuous, anonymous
assessments and delivering actionable insights to the HR.
\newline
\section{REQUIREMENTS}

\subsection{Functional Requirements}

\begin{itemize}
\item Schedule Survey
\end{itemize}

HR managers can schedule periodic mental health surveys
for employees via the platform. These surveys can be
customized based on different departments or roles. The
system ensures that surveys are sent automatically at pre-set
intervals, allowing organizations to consistently assess
employee well-being.
\newline
\begin{itemize}
    \item Fill Out Survey
    \end{itemize}
    
Employees will receive and complete mental health
surveys through the platform. These surveys are designed to
measure stress, burnout, and other mental health indicators.
The responses are processed securely to guarantee
confidentiality, ensuring employees feel comfortable
providing feedback.%
\newline

\par
\begin{itemize}
    \item Evaluate Mental Health Risk
    \end{itemize}
    
    The system will analyze survey responses using AI
    algorithms to assess mental health risks. Based on the data
    collected, the system will flag potential issues, such as burnout
    or high stress, and assign a risk level that HR professionals can
    review.
    \newline

\begin{itemize}
        \item Escalate to Psychologist
        \end{itemize}
        
        If the system identifies a high mental health risk, or if an
        HR manager requests further investigation, the case can be
        escalated to a psychologist. The psychologist will receive
        relevant reports from the system and can schedule follow-up
        appointments to offer individualized support.
        \newline

 \begin{itemize}
     \item Request Report
     \end{itemize}
            
     HR managers can generate detailed reports that highlight
     employee mental health trends. These reports provide
     aggregated data and insights into the overall well-being of the
     workforce, helping HR teams identify patterns and implement
     proactive measures.
\newline

\begin{itemize}
    \item Send Survey Completion Reminder
    \end{itemize}
           
    If employees fail to complete the survey within a set
timeframe, the HR managers can send reminders to encourage
participation. This ensures a higher completion rate and more
comprehensive data for mental health assessments.
\newline

\begin{itemize}
    \item Create Incident/Issue
    \end{itemize}
           
    Employees and their supervisors can report well-being-
related incidents or issues directly through the platform. This
feature allows them to highlight specific concerns or seek
help, which is then routed to HR or a psychologist based on
the severity of the issue.



\subsection{Non-functional Requirements}\label{AA}
The platform will prioritize the secure handling of
employee data by implementing encryption protocols to
ensure that all survey responses are stored and processed
securely. A key focus will be on maintaining anonymity and
confidentiality, allowing employees to trust that their personal
information is protected from unauthorized access. This
emphasis on data security will foster an environment of
openness, encouraging employees to participate honestly in
mental health assessments.\newline

Accessibility is also a critical feature of the platform. It
will be fully accessible on desktop devices, making it easy for
employees to participate in surveys and for HR teams to
review insights. The user interface will be intuitive and user-
friendly, reducing any friction in the user experience and
encouraging regular use by employees and HR professionals
alike. This approach ensures that the platform meets the needs
of a diverse workforce, fostering widespread adoption and
regular engagement.

\section{Technical Design}
The platform will be designed using a three-tier
architecture to ensure both scalability and efficiency. This
architecture is composed of three primary components: the
frontend, backend, and database.\newline

The frontend will consist of a web-based interface that
employees can use to complete mental health surveys, while
HR professionals will use the same interface to view insights
into overall employee well-being. The employee-facing
section of the interface will be minimalistic and simple,
emphasizing ease of use to encourage participation. In
contrast, the HR dashboard will offer rich data visualizations,
including charts, graphs, and filters, enabling HR teams to
explore mental health trends in depth and identify potential
areas of concern.\newline

The backend will contain the core application logic as well
as the AI engine responsible for analyzing survey data. This
AI engine will employ sophisticated algorithms to detect
patterns and trends, such as stress, burnout, and other mental
health risks. By processing data in real-time, the backend will
ensure the system operates smoothly and provides timely,
actionable insights.\newline

The database will securely store both raw survey responses
and the insights generated by the AI engine. A strong focus
will be placed on data security, with measures in place to
ensure confidentiality and protect sensitive information from
unauthorized access.\newline

For the User Interface, the employee-facing side will be
minimalistic, emphasizing simplicity to ensure that employees
can easily complete surveys without unnecessary distractions.
On the other hand, the HR dashboard will provide rich data
visualizations, including graphs, charts, and filters, allowing
HR professionals to explore the insights in detail. It will also
offer export options for reporting and analysis, making it a
powerful tool for understanding and managing employee
mental health.
\newline

\section{IMPLEMENTATION AND INTEGRATION}

\subsection {ASP.NET Core (Backend)}

    The project’s backend is implemented using ASP.NET Core, 
    a versatile and cross-platform framework ideal for building 
    robust, scalable RESTful APIs. Known for its high performance, 
    ASP.NET Core is compatible with both cloud and container-based 
    deployments, making it an excellent choice for handling the 
    demands of our mental health assessment platform. Its modular 
    architecture also allows for efficient maintenance and 
    integration of new features, which is beneficial as the 
    platform evolves. We will be using the newest LTR version, 
    .NET Core 8.0 for our backend application. The most common and 
    essential Libraries we are going to use, should already have 
    released new versions to support 8.0.
    \newline

\subsection {Python (AI Component)}

    Python is selected for implementing the AI algorithms 
    responsible for processing survey responses and 
    identifying mental health risks. Its extensive machine 
    learning and natural language processing libraries, 
    such as Scikit-learn, TensorFlow, and NLTK, make Python 
    well-suited for tasks that require sophisticated data 
    analysis. Python’s simplicity and readability are added 
    advantages, enabling the development of complex AI 
    models in a more manageable way, particularly for tasks 
    involving text analysis and sentiment detection within 
    survey responses.
    \newline


\subsection {JavaScript/TypeScript (Frontend)}

    To ensure cross-platform compatibility on Windows, Linux, 
    and macOS, the application's frontend will run in a web 
    browser. It will be built with JavaScript and TypeScript, 
    using the Vue.js framework alongside Bootstrap. TypeScript 
    is our primary language choice due to its strong typing 
    capabilities, which improve code quality, maintainability, 
    and scalability by reducing errors. This approach supports 
    a smooth experience for employees completing surveys and 
    for HR professionals analyzing data on the dashboard.
    \newline

\subsection {Docker Containerization}
    Our development team operates on multiple operating systems, 
    currently using both Windows and macOS. To streamline our 
    work across platforms and avoid common compatibility issues, 
    we decided to use Docker for our entire application, both in 
    deployment and development. Docker enables us to set up on 
    any development machine easily, without requiring installations 
    for .NET, Node/NPM, or SQL Server software. Using Docker is 
    especially beneficial for managing SQL databases, as it simplifies 
    handling ports and local connections that can otherwise be 
    problematic. Additionally, Docker provides us with flexible 
    deployment options for various server environments, allowing 
    us freedom in choosing a server provider. Most modern cloud 
    providers offer native support for Docker, and larger corporations 
    often provide their own managed Kubernetes clusters, allowing 
    us to run our Docker images seamlessly.

    Our Docker images will be Linux-based for speed and reliability, 
    and there are many prebuilt Docker images available that fit our 
    needs perfectly. We can use ready-made images for SQL Server and 
    deploy images that run our TypeScript or C# code with ease.
    \newline

\subsection {Cost estimation}
    
    Since this project is purely for educational purposes, we’ll stick 
    to software that is free for non-commercial use, avoiding the need 
    for licenses. To run our application and support a small number of 
    test users, we won’t require much computing power. For cloud hosting, 
    we estimate a budget of around \$5-20 per month.

\subsection {Software in use}

    We'll be using Docker to build and run our containers, installing the 
    Docker Desktop app on our development machines. To help complete the 
    project within our time limit, we’ll rely on several libraries.

    For the .NET Core backend API, we’ll use Entity Framework with the 
    code-first approach to simplify data storage integration. Additionally, 
    we’ll use OpenAPI Swagger, which provides an API standard and makes it 
    easier to test individual endpoints. For integrating our AI model, we’ll 
    need an additional library, though we haven’t chosen one yet.

    Our frontend will be built on the Vue.js framework, which includes several 
    built-in libraries. During development, we’ll use Vite to enable real-time 
    updates, allowing our changes to instantly reflect in the running application. 
    Naturally, we’ll also use TypeScript, which is technically external 
    software as well.

    \newline

    \subsection {Task distribution}

    The detailed task distribution for each team member will be 
    developed in the next project phase. This upcoming section 
    will allocate specific responsibilities and roles based 
    on the evolving requirements of the Mental Health Assessment 
    Tool. Each member’s tasks will align closely with the project 
    objectives, focusing on key aspects such as AI model 
    implementation, API development, frontend design, and backend 
    logic. By carefully distributing tasks according to each team 
    member’s expertise, we aim to ensure that all aspects of the 
    project are addressed with clear accountability and focus.
    \newline

    \section{SPECIFICATIONS}


    \subsection {Functional Requirements}
   
    \begin{itemize}
        \item Schedule Survey 
        \end{itemize}

    The Schedule Survey feature enables HR Managers to organize 
    regular mental health check-ins through scheduled surveys 
    tailored to specific departments or job roles. This feature 
    supports customization in frequency and target audience, 
    allowing surveys to be set up according to the needs of 
    different teams or positions within the organization. The 
    backend API in ASP.NET Core provides an endpoint for HR to 
    specify parameters such as survey frequency and targeted 
    departments. The backend also manages survey scheduling 
    logic by storing these configurations in the database, where 
    they are referenced to initiate automatic survey reminders. 
    Through a dedicated dashboard, HR Managers can define and 
    modify survey timing, frequency, and recipients as needed. 
    All survey schedules are documented in a schedules table 
    within the database, storing parameters like department IDs, 
    job roles, and frequency settings to keep track of survey 
    deployments and for audit purposes.
    \newline
   
    \begin{itemize}
        \item Fill Out Survey 
    \end{itemize}

    The Fill Out Survey function allows employees to complete 
    assigned mental health surveys securely and anonymously. 
    Employees receive surveys through the platform and can complete 
    them using a web-based, user-friendly interface. The frontend 
    form, developed in JavaScript or TypeScript, is designed for 
    ease of use and does not retain any personal information on 
    the client side, prioritizing data privacy. Upon submission, 
    responses are encrypted and processed by the ASP.NET Core 
    backend, ensuring that data is anonymized and securely stored. 
    The backend saves all survey responses in a responses table in 
    the database, which records each entry while maintaining 
    anonymity by linking responses only to anonymized identifiers.
    \newline    

    \begin{itemize}
        \item Evaluate Mental Health Risk
    \end{itemize}
    
    The Evaluate Mental Health Risk feature uses AI algorithms to 
    analyze employee responses, assessing their mental health status 
    and assigning risk levels based on survey answers. This automated
     analysis allows HR to quickly identify trends and potential 
     issues like stress or burnout. The AI component, implemented 
     in Python, applies machine learning models such as logistic 
     regression or decision trees to categorize responses by risk 
     level. Additionally, the platform uses natural language 
     processing (NLP) with libraries like NLTK or SpaCy to preprocess 
     text-based answers, performing sentiment analysis for deeper 
     insight into employee sentiment. The Python models are 
     accessible via a dedicated API, where the ASP.NET Core 
     backend calls these models after survey completion. Risk 
     assessments are stored in an evaluations table, providing a 
     structured record of mental health evaluations linked to 
     individual survey responses.
     \newline    

     \begin{itemize}
        \item Escalate to Psychologist
    \end{itemize}
    
    For high-risk cases, the Escalate to Psychologist feature 
    provides a pathway for further professional review, allowing 
    HR or the system itself to flag responses for psychologist 
    intervention. When a survey response is identified as high-risk, 
    this feature can escalate the case to an on-staff psychologist,
    ensuring prompt follow-up. The backend includes an API endpoint 
    for escalation requests, which triggers an automatic 
    notification to the psychologist through email or in-app 
    alerts. Details of each escalated case, including the response
    ID, psychologist ID, and relevant timestamps, are stored in an
    escalations table within the database. This feature ensures 
    timely intervention and follow-up on mental health concerns
    flagged by the system.
    \newline    

     \begin{itemize}
        \item Request Report
    \end{itemize}
    

    The Request Report feature provides HR Managers with a 
    comprehensive view of employee mental health trends. The 
    reporting dashboard offers visual summaries through charts, 
    graphs, and other graphical data presentations, providing HR
     with insights into overall well-being within their teams. 
     Developed with tools like Chart.js or D3.js, the dashboard 
     allows HR to explore trends over time and filter data based 
     on department or risk level. The backend aggregates survey 
     data to produce these insights in real-time, with queries on 
     the responses and evaluations tables to generate trends that 
     are accessible and actionable for HR analysis.
     \newline    

     \begin{itemize}
        \item Send Survey Completion Reminder
    \end{itemize}
    
    The Send Survey Completion Reminder feature improves 
    participation rates by notifying employees who have not 
    completed their surveys within a specified timeframe. The 
    backend includes a scheduled job that periodically checks 
    for incomplete survey entries and automatically sends 
    reminders. Notifications are dispatched via email or in-app 
    alerts to encourage timely survey completion, helping HR gather 
    accurate, comprehensive data. In the database, completed surveys 
    are flagged in the responses table, while any incomplete entries
    trigger reminders, thus maintaining high response rates and 
    supporting data reliability.
    \newline    

     \begin{itemize}
        \item Create Incident/Issue
    \end{itemize}
    
    The Create Incident/Issue feature enables employees or supervisors 
    to report well-being-related concerns directly on the platform. 
    Through a straightforward form on the frontend, users can report 
    issues, specifying details like incident description, employee ID, 
    and severity level. The backend routes these reports based on their 
    severity to either HR or a psychologist, ensuring each issue is 
    addressed promptly. Incident details are stored in an incidents
    table in the database, providing a structured record of all 
    reported issues. This system ensures that employee concerns are 
    tracked, documented, and addressed according to severity and 
    relevance
    \newline    

    \subsection {Non-Functional Requirements}
   
    \begin{itemize}
        \item Data Security and Privacy
        \end{itemize}
    
    Data Security and Privacy are critical components of the platform, 
    ensuring that all sensitive information is protected throughout the 
    survey process. The platform uses advanced encryption methods like 
    AES or RSA to secure data-at-rest and TLS for data-in-transit, 
    ensuring confidentiality and data protection at all stages. 
    Role-based access control further secures sensitive information, 
    permitting only authorized personnel, such as HR managers and 
    psychologists, to view or manage confidential data. Compliance with
    standards like GDPR protects employee anonymity, promoting trust 
    and encouraging participation in surveys.
    \newline    

    \begin{itemize}
        \item Scalability
        \end{itemize}
    
    Scalability is achieved through a three-tier architecture, where the 
    frontend, backend, and database layers work independently but in 
    unison. This architecture enables the system to scale efficiently 
    and accommodate growing numbers of users or data loads. To manage 
    higher concurrent usage, the platform also supports cloud 
    infrastructure options on services like AWS or GCP and can be 
    deployed in Docker containers for modular expansion. These 
    load-balancing and scaling capabilities ensure that the platform 
    remains responsive and effective as the organization expands or as 
    usage increases.
    \newline    

    \begin{itemize}
        \item Accessibility
        \end{itemize}

    The Accessibility of the platform is designed with a responsive, 
    cross-platform interface that supports a range of devices, from 
    Desktops to tablets and mobile phones. The user interface is built 
    to be user-friendly, accommodating employees of all technical 
    backgrounds to maximize engagement and ease of use. By supporting 
    multiple device types and offering a simplified, intuitive user 
    experience, the platform ensures high levels of accessibility, 
    facilitating smooth participation in surveys and effective 
    engagement with the HR dashboard
    \newline  \newline  \newline  
    
    \href{https://github.com/Creebos/Mindwatch.git}{https://github.com/Creebos/Mindwatch.git}.

\end{document}
