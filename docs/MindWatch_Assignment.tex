\documentclass[conference]{IEEEtran}
\IEEEoverridecommandlockouts

\usepackage{cite}
\usepackage{amsmath,amssymb,amsfonts}
\usepackage{algorithmic}
\usepackage{graphicx}
\usepackage{textcomp}
\usepackage{xcolor}
\def\BibTeX{{\rm B\kern-.05em{\sc i\kern-.025em b}\kern-.08em
    T\kern-.1667em\lower.7ex\hbox{E}\kern-.125emX}}
\begin{document}

\title{MindWatch: Keeping Tech Minds in Balance\\
{\footnotesize \textsuperscript{}}
}

\author{\IEEEauthorblockN{Riccardo Dennis William}
\IEEEauthorblockA{\textit{Department of Information System} \\
\textit{Hanyang University}\\
Seoul, South Korea \\
riccaden@students.zhaw.ch}
\and
\IEEEauthorblockN{Meyer Yves}
\IEEEauthorblockA{\textit{Department of Information System} \\
\textit{Hanyang University}\\
Seoul, South Korea \\
meyeryve@students.zhaw.ch}
\and
\IEEEauthorblockN{Wu Yunjie}
\IEEEauthorblockA{\textit{Department of Information System} \\
\textit{Hanyang University}\\
Seoul, South Korea \\
19855420423@163.com}
\and
\IEEEauthorblockN{Pereira Leandro}
\IEEEauthorblockA{\textit{Department of Information System} \\
\textit{Hanyang University}\\
Seoul, South Korea \\
pereilea@students.zhaw.ch}
}

\maketitle

\begin{abstract}
    This project aims to develop a comprehensive web-based
    platform designed to assess the mental health of employees in
    the tech industry through AI-driven analysis of survey
    responses and performance data. The platform will provide
    valuable insights into employee well-being, functioning as an
    early warning system to detect potential mental health concerns
    before they escalate. By offering a dashboard tailored for HR
    professionals, the system enables data-driven decision-making,
    allowing organizations to proactively support employee mental
    health. A key aspect of the platform is the secure handling of
    survey data, ensuring anonymity and confidentiality, which
    fosters trust and encourages widespread adoption among
    employees.
\end{abstract}

\begin{IEEEkeywords}
    mental health, AI-driven analysis, employee well-being, tech industry, HR dashboard, data security, survey automation, burnout detection, stress management, predictive analytics, human resources, data-driven decision-making.
\end{IEEEkeywords}


\begin{table}[htbp]
    \caption{Role Assignments (First half of Project)}
    \centering
    \begin{tabular}{|p{1.8cm}|p{0.9cm}|p{4.8cm}|}
        \hline
        \textbf{Roles} & \textbf{Name} & \textbf{Tasks} \\ \hline
        User & Yunjie & Explaining use-cases and testing demo \\ \hline
        Customer & Leandro & Change use-cases during development \\ \hline
        Software Developer & Leandro, Yunjie, Yves, Dennis & Implementing software and creating model \\ \hline
        Development Manager & Yves, Dennis & Prioritize tasks and organize teams \\ \hline
    \end{tabular}
    \label{tab:role_assignments}
\end{table}


\section{Introduction}
The tech industry is known for its high-pressure work environment, 
often leading to mental health challenges among employees. 
Early identification of potential mental health risks and timely 
support can significantly improve employee well-being and overall 
productivity. However, many organizations lack consistent, 
confidential tools to monitor and assess mental health across their 
workforce. This project seeks to address this gap by developing a 
scalable, AI-powered solution for continuous mental health monitoring.

Tech professionals frequently face long hours, tight
deadlines, and high-stakes projects, all of which can
contribute to stress, burnout, and other mental health issues.
Without proper monitoring tools, organizations risk higher
turnover rates, decreased productivity, and rising healthcare
costs. The client, LG, requires a system that will enable HR
departments to identify early signs of mental health challenges
before they escalate into more severe problems.\newline

While there are existing platforms such as Headspace and
Calm, which focus on mental well-being, few solutions offer
continuous mental health monitoring specifically tailored to
the tech industry. Furthermore, many available tools lack
integration with company data or actionable insights for HR
professionals to drive meaningful interventions. Our platform
aims to fill this gap by providing continuous, anonymous
assessments and delivering actionable insights to the HR.

\section{REQUIREMENTS}

\subsection{Functional Requirements}

\begin{itemize}
\item Schedule Survey
\end{itemize}

HR managers can schedule periodic mental health surveys
for employees via the platform. These surveys can be
customized based on different departments or roles. The
system ensures that surveys are sent automatically at pre-set
intervals, allowing organizations to consistently assess
employee well-being.
\newline
\begin{itemize}
    \item Fill Out Survey
    \end{itemize}
    
Employees will receive and complete mental health
surveys through the platform. These surveys are designed to
measure stress, burnout, and other mental health indicators.
The responses are processed securely to guarantee
confidentiality, ensuring employees feel comfortable
providing honest feedback.%
\newline

\par
\begin{itemize}
    \item Evaluate Mental Health Risk
    \end{itemize}
    
    The system will analyze survey responses using AI
    algorithms to assess mental health risks. Based on the data
    collected, the system will flag potential issues, such as burnout
    or high stress, and assign a risk level that HR professionals can
    review.
    \newline

\begin{itemize}
        \item Escalate to Psychologist
        \end{itemize}
        
        If the system identifies a high mental health risk, or if an
        HR manager requests further investigation, the case can be
        escalated to a psychologist. The psychologist will receive
        relevant reports from the system and can schedule follow-up
        appointments to offer individualized support.
        \newline

 \begin{itemize}
     \item Request Report
     \end{itemize}
            
     HR managers can generate detailed reports that highlight
     employee mental health trends. These reports provide
     aggregated data and insights into the overall well-being of the
     workforce, helping HR teams identify patterns and implement
     proactive measures.
\newline

\begin{itemize}
    \item Send Survey Completion Reminder
    \end{itemize}
           
    If employees fail to complete the survey within a set
timeframe, the HR managers can send reminders to encourage
participation. This ensures a higher completion rate and more
comprehensive data for mental health assessments.
\newline

\begin{itemize}
    \item Create Incident/Issue
    \end{itemize}
           
    Employees and their supervisors can report well-being-
related incidents or issues directly through the platform. This
feature allows them to highlight specific concerns or seek
help, which is then routed to HR or a psychologist based on
the severity of the issue.



\subsection{Non-functional Requirements}\label{AA}
The platform will prioritize the secure handling of
employee data by implementing encryption protocols to
ensure that all survey responses are stored and processed
securely. A key focus will be on maintaining anonymity and
confidentiality, allowing employees to trust that their personal
information is protected from unauthorized access. This
emphasis on data security will foster an environment of
openness, encouraging employees to participate honestly in
mental health assessments.\newline

Accessibility is also a critical feature of the platform. It
will be fully accessible on desktop devices, making it easy for
employees to participate in surveys and for HR teams to
review insights. The user interface will be intuitive and user-
friendly, reducing any friction in the user experience and
encouraging regular use by employees and HR professionals
alike. This approach ensures that the platform meets the needs
of a diverse workforce, fostering widespread adoption and
regular engagement.

\section{Technical Design}
The platform will be designed using a three-tier
architecture to ensure both scalability and efficiency. This
architecture is composed of three primary components: the
frontend, backend, and database.\newline

The frontend will consist of a web-based interface that
employees can use to complete mental health surveys, while
HR professionals will use the same interface to view insights
into overall employee well-being. The employee-facing
section of the interface will be minimalistic and simple,
emphasizing ease of use to encourage participation. In
contrast, the HR dashboard will offer rich data visualizations,
including charts, graphs, and filters, enabling HR teams to
explore mental health trends in depth and identify potential
areas of concern.\newline

The backend will contain the core application logic as well
as the AI engine responsible for analyzing survey data. This
AI engine will employ sophisticated algorithms to detect
patterns and trends, such as stress, burnout, and other mental
health risks. By processing data in real-time, the backend will
ensure the system operates smoothly and provides timely,
actionable insights.\newline

The database will securely store both raw survey responses
and the insights generated by the AI engine. A strong focus
will be placed on data security, with measures in place to
ensure confidentiality and protect sensitive information from
unauthorized access.\newline

For the User Interface, the employee-facing side will be
minimalistic, emphasizing simplicity to ensure that employees
can easily complete surveys without unnecessary distractions.
On the other hand, the HR dashboard will provide rich data
visualizations, including graphs, charts, and filters, allowing
HR professionals to explore the insights in detail. It will also
offer export options for reporting and analysis, making it a
powerful tool for understanding and managing employee
mental health.


\section{IMPLEMENTATION AND INTEGRATION}
For AI-driven analysis, the platform will leverage
Python, a highly versatile and powerful programming
language known for its extensive libraries and frameworks
dedicated to machine learning and data analysis. Python’s AI
libraries will be instrumental in building the machine
learning models that will analyze survey responses, detect
mental health trends, and provide predictive insights.
\newline

For the backend and the creation of the RESTful web
API, ASP.NET Core will be used. ASP.NET Core is a cross-
platform, high-performance framework known for building
robust, scalable, and secure web applications. 
\newline

For the frontend, JavaScript and TypeScript will be
employed. JavaScript, the most widely used programming
language for web development, will ensure a dynamic,
responsive user experience. TypeScript, a superset of
JavaScript, adds static types, enabling better code quality,
scalability, and easier debugging during development






\end{document}
